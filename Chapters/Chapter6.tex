\chapter{Conclusion and Future Work}

\label{Chapter6}

In this final chapter, we summarize the key contributions of this thesis and reflect on the outcomes of the research. We presented an AI-driven medical chatbot designed to serve both medical students and patients, integrating advanced natural language processing techniques, embedding-based retrieval, and large language models (LLMs) into a unified system. Our approach combines a robust data ingestion pipeline with role-based interaction, ensuring that each user receives tailored, context-aware responses. 

\section{Conclusion}
This thesis has demonstrated a novel framework  conversational AI and agentic AI in the medical domain by addressing critical challenges in both educational and patient support contexts. The main contributions can be summarized as follows:
\begin{itemize}
    \item \textbf{Dual-Mode Chatbot Architecture:} We designed a system that distinguishes between medical students and patients through secure user authentication, role assignment, and role-based logic. This allows the chatbot to deliver in-depth academic explanations to students while providing concise, practical guidance to patients.
    \item \textbf{Robust Data Processing Pipeline:} The implementation includes comprehensive PDF processing, where documents are efficiently parsed, chunked, and preprocessed to facilitate accurate embedding generation.
    \item \textbf{Embedding and Vector Storage:} By leveraging state-of-the-art models (such as \texttt{all-mpnet-base-v2}) and integrating with the Pinecone vector database, the system achieves rapid and scalable retrieval of contextually relevant text.
    \item \textbf{Retrieval-Augmented Generation (RAG):} The RAG pipeline effectively combines retrieved text from the vector database with LLM-based response generation. This integration mitigates the risk of hallucination and grounds the chatbot's answers in factual, verifiable sources.
    \item \textbf{Comprehensive Evaluation:} Both quantitative metrics (e.g., cosine similarity, Silhouette Score) and qualitative human evaluations were conducted. The results show that our approach provides high semantic alignment between expected and generated responses, validating the efficacy of the system.
\end{itemize}

\section{Answers to Research Questions}
The research questions posed at the outset of this thesis have been addressed as follows:
\begin{enumerate}
    \item \textbf{Supporting Medical Education:} The chatbot provides dynamic, evidence-based answers that bridge theoretical knowledge and practical scenarios, thereby enhancing learning for medical students.
    \item \textbf{Improving Patient Guidance:} By employing symptom analysis and retrieval-augmented generation, the system offers clear and contextually appropriate medical advice, assisting patients in deciding whether to seek routine or urgent care.
    \item \textbf{Healthcare Provider Recommendations:} The integration of location- and specialty-based filters enables the system to suggest suitable healthcare providers, thereby streamlining patient referrals.
    \item \textbf{Autonomous Appointment Management:} The agentic component successfully automates appointment booking by integrating with external APIs, reducing administrative burden and ensuring timely patient care.
\end{enumerate}

\section{Future Work and Challenges}
While the proposed system has shown promising results, several challenges remain and avenues for future research include:
\begin{itemize}
    \item \textbf{Handling Data Noise:} Despite effective preprocessing, some residual noise remains in the data. Future work could explore advanced noise-reduction techniques to further refine input quality.
    \item \textbf{Scalability and Efficiency:} With the current computational resources, certain models had to be downscaled. As more powerful hardware becomes available, future iterations could employ larger, more sophisticated models to improve performance.
    \item \textbf{Multimodal Data Integration:} Incorporating additional data sources (such as medical imaging or sensor data) could provide a more comprehensive understanding of patient conditions and enhance the chatbot’s recommendations.
    \item \textbf{Enhanced Prompt Engineering:} Further research into prompt optimization for the LLM may reduce hallucinations and improve the accuracy of generated responses.
    \item \textbf{User Interface Improvements:} Refining the front-end experience—both in the Flask templates and the Chainlit chat UI—could lead to higher user satisfaction and better overall interaction.
\end{itemize}
\clearpage
