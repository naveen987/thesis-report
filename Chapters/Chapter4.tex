\chapter{Implementation}

\label{Chapter4}

\lhead{Chapter 4. \emph{Implementation}}

In this chapter, a comprehensive overview of the libraries utilized throughout the project is presented, 
alongside the roles each library plays in different sections of the code. By offering deeper insight 
into how each component works and how the algorithms are implemented, this thesis is brought to its conclusion.

For the implementation of the thesis, I have used platforms such as VS Code (Visual Studio Code) and Jupyter Notebook. 
While VS Code is an integrated development environment (IDE) suitable for a variety of languages and extensions, 
Jupyter Notebook offers a web-based environment that is well-suited for interactive coding and data visualization.

\vspace{1cm}
%%%%%%%%%%%%%%%%%%%%%%%%%%%%%%%%%%%%%%%%
\section{Libraries}
%%%%%%%%%%%%%%%%%%%%%%%%%%%%%%%%%%%%%%%%
In this section, we focus on the various libraries used in the project and how they contribute to the overall model implementation process.

\vspace{0.4cm}
%%%%%%%%%% sqlite3 %%%%%%%%%%
\noindent\textbf{sqlite3:}

\noindent
The \texttt{sqlite3} module is part of Python’s standard library, enabling lightweight database operations. 
It is used here to store and retrieve user information, roles, and session data. This is particularly helpful 
for managing authentication details in a local, file-based database, without requiring an external database server \cite{pythonlibrary}.

\vspace{0.4cm}
%%%%%%%%%% httpcore %%%%%%%%%%
\noindent\textbf{httpcore (TimeoutException):}

\noindent
\texttt{httpcore} is a low-level HTTP library used to manage connection pooling, timeouts, and retries. 
In this thesis, it is mainly leveraged to handle \texttt{TimeoutException} cases, ensuring the application 
does not hang indefinitely during network calls \cite{pythonlibrary}.

\vspace{0.4cm}
%%%%%%%%%% Selenium %%%%%%%%%%
\noindent\textbf{Selenium:}

\noindent
Selenium automates browsers, enabling dynamic web scraping. It is used in conjunction with the Edge WebDriver
to navigate pages, locate elements, and extract data. Relevant classes such as \texttt{Service}, \texttt{Options}, 
and \texttt{WebDriverWait} help in configuring and controlling the browser for scraping tasks \cite{pythonlibrary}.

\vspace{0.4cm}
%%%%%%%%%% BeautifulSoup %%%%%%%%%%
\noindent\textbf{BeautifulSoup (bs4):}

\noindent
\texttt{BeautifulSoup} parses HTML (and XML) content, transforming the raw webpage data into a navigable tree. 
Used together with Selenium, it allows the program to extract specific tags, links, or other structured elements 
from the loaded pages \cite{pythonlibrary}.

\vspace{0.4cm}
%%%%%%%%%% time %%%%%%%%%%
\noindent\textbf{time:}

\noindent
This built-in Python module provides sleep intervals and basic time management. Incorporating small delays can prevent 
overwhelming target servers, especially when performing repetitive automated scraping \cite{pythonlibrary}.

\vspace{0.4cm}
%%%%%%%%%% json %%%%%%%%%%
\noindent\textbf{json:}

\noindent
The \texttt{json} module encodes and decodes data in JSON (JavaScript Object Notation). It is both human-readable
and language-independent, making it ideal for transmitting and storing structured data. In this project, JSON 
formats are used to manage intermediate results, user sessions, or to pass data between components \cite{pythonlibrary}.

\vspace{0.4cm}
%%%%%%%%%% chainlit %%%%%%%%%%
\noindent\textbf{chainlit:}

\noindent
\texttt{chainlit} provides a framework for building chat-based interfaces around large language models (LLMs). 
It handles session management, real-time conversation flows, and prompt configuration, giving users an 
intuitive platform to interact with AI-driven functionalities \cite{pythonlibrary}.

\vspace{0.4cm}
%%%%%%%%%% src.helper (download_hugging_face_embeddings) %%%%%%%%%%
\noindent\textbf{src.helper:}

\noindent
While not a standard library, this local module contains custom utilities such as \texttt{download\_hugging\_face\_embeddings}. 
It automates loading the required embeddings from Hugging Face, ensuring consistent embedding generation 
across the system \cite{pythonlibrary}.

\vspace{0.4cm}
%%%%%%%%%% langchain_community.vectorstores (Pinecone) %%%%%%%%%%
\noindent\textbf{langchain\_community.vectorstores (Pinecone):}

\noindent
This integration connects LangChain’s data processing workflows with the Pinecone vector database. By creating embeddings 
and storing them in Pinecone, the system retrieves similar text chunks or documents quickly, enhancing response accuracy 
in the chatbot \cite{pythonlibrary}.

\vspace{0.4cm}
%%%%%%%%%% pinecone (PineconeClient, ServerlessSpec) %%%%%%%%%%
\noindent\textbf{pinecone:}

\noindent
The \texttt{pinecone} SDK communicates with the Pinecone service, a specialized vector database built for high-speed 
similarity search. \texttt{PineconeClient} handles index creation and queries, while \texttt{ServerlessSpec} configures 
the underlying infrastructure \cite{pythonlibrary}.

\vspace{0.4cm}
%%%%%%%%%% langchain.prompts (PromptTemplate) %%%%%%%%%%
\noindent\textbf{langchain.prompts (PromptTemplate):}

\noindent
\texttt{PromptTemplate} structures how queries are fed into large language models. By separating static and dynamic 
parts of a prompt, it creates systematic instructions for LLM-based generation \cite{pythonlibrary}.

\vspace{0.4cm}
%%%%%%%%%% langchain_community.llms (CTransformers) %%%%%%%%%%
\noindent\textbf{langchain\_community.llms (CTransformers):}

\noindent
\texttt{CTransformers} provides an interface for loading and running transformer models within the LangChain ecosystem. 
It can be more memory-efficient and can help speed up inference in certain configurations \cite{pythonlibrary}.

\vspace{0.4cm}
%%%%%%%%%% langchain.chains (RetrievalQA) %%%%%%%%%%
\noindent\textbf{langchain.chains (RetrievalQA):}

\noindent
\texttt{RetrievalQA} coordinates the search of relevant vectors/documents before the final language model call. 
This leads to more accurate, context-specific responses to user questions, making the chatbot more robust \cite{pythonlibrary}.

\vspace{0.4cm}
%%%%%%%%%% langchain.document_loaders (PyPDFLoader) %%%%%%%%%%
\noindent\textbf{langchain.document\_loaders (PyPDFLoader):}

\noindent
\texttt{PyPDFLoader} helps parse PDF documents into text chunks, enabling easy ingestion of content into the retrieval 
system. This is crucial when dealing with PDF-based data sets \cite{pythonlibrary}.

\vspace{0.4cm}
%%%%%%%%%% langchain.text_splitter (CharacterTextSplitter) %%%%%%%%%%
\noindent\textbf{langchain.text\_splitter (CharacterTextSplitter):}

\noindent
Splits text into manageable chunks based on character count. By doing so, the system can embed and retrieve smaller, 
more relevant pieces of information during user queries \cite{pythonlibrary}.

\vspace{0.4cm}
%%%%%%%%%% dotenv (load_dotenv) %%%%%%%%%%
\noindent\textbf{dotenv (load\_dotenv):}

\noindent
Loads environment variables from a \texttt{.env} file, preventing sensitive credentials (like API keys) from 
being exposed in the source code. This approach enhances security and deployment flexibility \cite{pythonlibrary}.

\vspace{0.4cm}
%%%%%%%%%% sentence_transformers (SentenceTransformer) %%%%%%%%%%
\noindent\textbf{sentence\_transformers:}

\noindent
This library provides state-of-the-art models for sentence and paragraph embeddings. By embedding text into vector 
representations, the system can compare semantic similarities and rank documents according to relevance \cite{pythonlibrary}.

\vspace{0.4cm}
%%%%%%%%%% twilio.rest (Client) %%%%%%%%%%
\noindent\textbf{twilio.rest (Client):}

\noindent
Twilio’s Python SDK handles sending SMS messages or other communications. The chatbot uses this to send appointment 
reminders or booking confirmations to users, enhancing the system’s real-world utility \cite{pythonlibrary}.

\vspace{0.4cm}
%%%%%%%%%% os %%%%%%%%%%
\noindent\textbf{os:}

\noindent
A built-in Python module for OS-level interactions, such as file path handling and environment variable lookups. 
Used to coordinate local file structure and manage \texttt{.env} variables \cite{pythonlibrary}.

\vspace{0.4cm}
%%%%%%%%%% logging %%%%%%%%%%
\noindent\textbf{logging:}

\noindent
\texttt{logging} is the standard Python library for capturing and recording logs, warnings, and error messages. 
It helps in debugging and monitoring the application’s performance \cite{pythonlibrary}.

\vspace{0.4cm}
%%%%%%%%%% warnings %%%%%%%%%%
\noindent\textbf{warnings:}

\noindent
The \texttt{warnings} module is used to handle non-critical warnings in Python. In this context, it can hide or filter 
out messages from dependencies that are not relevant to the end-user \cite{pythonlibrary}.

\vspace{0.4cm}
%%%%%%%%%% asyncio %%%%%%%%%%
\noindent\textbf{asyncio:}

\noindent
\texttt{asyncio} facilitates concurrent execution of tasks, leveraging an event loop to handle long-running or 
I/O-bound processes without blocking other functionalities. Particularly useful for background tasks in chatbots \cite{pythonlibrary}.

\vspace{0.4cm}
%%%%%%%%%% flask %%%%%%%%%%
\noindent\textbf{Flask:}

\noindent
Flask is a lightweight Python framework used to build web applications and RESTful APIs. 
It manages server-side processes, routes, and user sessions, playing an important role in bridging
the front-end user interface with back-end logic \cite{pythonlibrary}.

\vspace{0.4cm}
%%%%%%%%%% database (init_db, get_user, add_user) %%%%%%%%%%
\noindent\textbf{database:}

\noindent
This local Python module contains functions like \texttt{init\_db}, \texttt{get\_user}, and \texttt{add\_user}, 
supporting user registration, retrieval, and management. It leverages \texttt{sqlite3} to store and fetch user credentials \cite{pythonlibrary}.

\vspace{0.4cm}
%%%%%%%%%% subprocess %%%%%%%%%%
\noindent\textbf{subprocess:}

\noindent
The \texttt{subprocess} module lets you spawn new processes, connect to their input/output/error pipes,
and obtain their return codes. This functionality is essential when starting or stopping auxiliary services 
like a Chainlit server \cite{pythonlibrary}.

\vspace{0.4cm}
%%%%%%%%%% threading %%%%%%%%%%
\noindent\textbf{threading:}

\noindent
\texttt{threading} supports concurrency by running multiple threads in parallel. In this project, it helps 
avoid blocking the main application by delegating tasks (such as server checks) to background threads \cite{pythonlibrary}.

\vspace{0.4cm}
%%%%%%%%%% requests %%%%%%%%%%
\noindent\textbf{requests:}

\noindent
\texttt{requests} simplifies making HTTP requests. It is used to communicate with external APIs (e.g., Pinecone, 
Twilio, or third-party services) and verify that local services like Chainlit are up and running \cite{pythonlibrary}.

\vspace{0.4cm}
%%%%%%%%%% numpy %%%%%%%%%%
\noindent\textbf{numpy:}

\noindent
\texttt{numpy} provides a multi-dimensional array object and routines for fast array manipulations, 
serving as the fundamental package for numerical computations in Python \cite{pythonlibrary}.

\vspace{0.4cm}
%%%%%%%%%% plotly.graph_objects %%%%%%%%%%
\noindent\textbf{plotly.graph\_objects:}

\noindent
\texttt{plotly} is a plotting library that generates interactive data visualizations. 
The \texttt{graph\_objects} module provides granular control over chart elements, 
suitable for in-depth analysis of chatbot performance data \cite{pythonlibrary}.

\vspace{1cm}
\noindent\textbf{HTML/CSS:}

The front-end user interface is structured using HTML and styled with CSS. HTML forms gather user input, 
while CSS ensures the layout is visually appealing and user-friendly. This combo supports usability 
and meets the requirements for practical deployment.

%%%%%%%%%% Code Implementation %%%%%%%%%%%

\section{Code Implementation}

In this chapter, we map the major steps of our **architecture** (see Figure~\ref{fig:system-architecture}) to their respective **code components**. Each section follows the flow of data and logic from user authentication all the way to generating a final answer.

\section{User Authentication and Role Identification}
\label{sec:auth-role}
User management begins with:
\begin{itemize}
    \item \textbf{Signup \& Role Assignment:} A new user provides an email and password. The system assigns the role (\textit{e.g., patient} or \textit{medical student}) by storing it in a lightweight database (e.g., using \texttt{sqlite3}).
    \item \textbf{Login:} On login, the code checks the role from the database to serve the correct chatbot mode.
    \item \textbf{Session Handling:} Flask or Chainlit stores session data (e.g., user email and role) so each request can correctly route to patient or student logic.
\end{itemize}

\section{PDF Processing (Text Extraction and Chunking)}
\label{sec:pdf-chunking}
PDF documents (e.g., medical articles) are uploaded by students:
\begin{itemize}
    \item \textbf{Extraction:} The code typically uses a library like \texttt{PyPDFLoader} to read and convert PDF pages to raw text.
    \item \textbf{Chunking:} Long text is split into manageable pieces via \texttt{CharacterTextSplitter} or custom splitting logic, ensuring each chunk remains semantically coherent. 
    \item \textbf{Preprocessing:} Code may further remove stopwords or apply cleaning to each chunk before embedding.
\end{itemize}

\section{Embedding Generation}
\label{sec:embedding}
Each text chunk is converted into a numerical representation (an \textit{embedding}):
\begin{itemize}
    \item \textbf{Model Setup:} A pretrained model (e.g., \texttt{all-mpnet-base-v2} or \texttt{SentenceTransformer}) is loaded.
    \item \textbf{Batch Embedding:} The code loops through chunks and feeds them into the embedding model. This step often includes concurrency or batch processing to optimize runtime.
    \item \textbf{Storage:} Embeddings are passed either to an in-memory structure or a vector database (such as Pinecone) for fast retrieval.
\end{itemize}

\section{Retrieval-Augmented Generation Pipeline}
\label{sec:rag-pipeline}
To handle user queries, we blend retrieval (based on embeddings) with LLM generation:
\begin{itemize}
    \item \textbf{Query Handling:} Once a user query arrives, the system transforms it into an embedding (using the same model as above).
    \item \textbf{Similarity Search:} This embedding is compared against stored embeddings in Pinecone (or another vector store). The top-$k$ relevant chunks are returned.
    \item \textbf{Prompt Augmentation:} These chunks are appended to the user’s query to form a “context window” for the Large Language Model (LLM).
    \item \textbf{Response Generation:} The LLM (e.g., Mistral 13B) produces an answer grounded in the retrieved text. The code also applies any prompt template rules or role-specific instructions.
\end{itemize}

\section{Role-Based Logic (Learning Assistant vs. Diagnostic Agent)}
\label{sec:role-logic}
In the code, a conditional block or chain-level branching typically handles:
\begin{itemize}
    \item \textbf{Student Mode (Learning AI Assistant)}:  
    Provides detailed, in-depth explanations with references.
    \item \textbf{Patient Mode (Agentic Diagnostic AI)}:  
    Focuses on simpler language, symptom triage, or booking logic. 
    When users mention booking an appointment, an agentic step can automatically interact with a scheduling API.
\end{itemize}

\section{Symptom Analysis, Provider Recommendation, and Appointment Booking}
\label{sec:extended-capabilities}
For patient-focused tasks, code sections may:
\begin{itemize}
    \item \textbf{Symptom Analysis}: Use a rule-based approach or LLM prompts fine-tuned on medical data to interpret user symptoms.
    \item \textbf{Provider Recommendation}: Parse a local database (or external service) of doctors. The code filters by specialty or location, returning suitable recommendations.
    \item \textbf{Appointment Booking}: Integrate with an external API (e.g., Twilio for SMS, Doctolib for scheduling). The code triggers a request, confirms details, and notifies the user.
\end{itemize}

\section{Front-End Integration}
\label{sec:frontend}
\begin{itemize}
    \item \textbf{Flask or Chainlit Routes}: A Python route handles file uploads (for PDFs) or query input. These routes pass data to the back-end logic (embedding, retrieval, LLM).
    \item \textbf{User Interface}: HTML/CSS or a Chainlit-based chat bubble interface. The code updates the chat context in real-time and displays responses from the LLM.
    \item \textbf{Templating}: If using Flask, Jinja templates might render dynamic pages with user-specific data.
\end{itemize}

\section{Error Handling and Logging}
\label{sec:error-logging}
Throughout the system:
\begin{itemize}
    \item \textbf{Try-Except Blocks}: Key places (e.g., PDF loading, retrieval queries, LLM calls) are wrapped in \texttt{try-except} to gracefully handle failures.
    \item \textbf{Logging}: The Python \texttt{logging} module or a custom logger records warnings, debug info, and errors. This helps in diagnosing issues without exposing them to end users.
    \item \textbf{Notifications}: In critical scenarios (like an inability to retrieve from the vector database), the code might send an alert or fallback to a simpler pipeline.
\end{itemize}

\section{User Authentication and Role Identification}
\label{sec:auth-role}

Below is an excerpt of Python/Flask code illustrating how user login, signup, and role identification are handled:

\begin{lstlisting}[language=Python, caption={User authentication and role assignment code}, basicstyle=\small\ttfamily]
from flask import Flask, render_template, request, redirect, url_for
from database import init_db, get_user, add_user
import subprocess
import threading
import os
import time
import requests

app = Flask(__name__)

# Initialize the database
init_db()

FLASK_PORT = 5002
CHATBOT_PORT = 8001
chainlit_process = None

def is_chainlit_running():
    try:
        response = requests.get(f"http://localhost:{CHATBOT_PORT}")
        return response.status_code == 200
    except requests.exceptions.ConnectionError:
        return False

def start_chainlit(email):
    global chainlit_process
    if is_chainlit_running():
        print("Chainlit is already running.")
        return

    print(f"Launching Chainlit on port {CHATBOT_PORT}...")
    chainlit_process = subprocess.Popen(
        ["chainlit", "run", "app.py", "--port", str(CHATBOT_PORT)],
        cwd=os.path.dirname(os.path.abspath(__file__)),
        stdout=subprocess.DEVNULL,
        stderr=subprocess.DEVNULL,
        start_new_session=True,
        env={
            **os.environ,
            "LOGGED_IN_EMAIL": email
        }
    )

@app.route("/", methods=["GET", "POST"])
def login():
    """Login route to authenticate users and start Chainlit."""
    if request.method == "POST":
        email = request.form.get("email")
        password = request.form.get("password")
        user = get_user(email)
        if user and user[2] == password:
            print(f"LOGIN SUCCESS: {email}")

            # Start Chainlit if not running; pass the email to the new process
            threading.Thread(target=start_chainlit, args=(email,), daemon=True).start()

            # After login, display a loading page that will link to Chainlit
            chatbot_url = f"http://localhost:{CHATBOT_PORT}/"
            print(f"REDIRECTING TO: {chatbot_url}")
            return render_template("loading.html", chatbot_url=chatbot_url)
        else:
            return render_template("login.html", error="Invalid email or password.")
    return render_template("login.html")

@app.route("/signup", methods=["GET", "POST"])
def signup():
    """Signup route to register new users."""
    if request.method == "POST":
        email = request.form.get("email")
        password = request.form.get("password")
        role = request.form.get("role")
        if add_user(email, password, role):
            return redirect(url_for("login"))
        else:
            return render_template("signup.html", error="Email already registered.")
    return render_template("signup.html")

if __name__ == "__main__":
    app.run(port=FLASK_PORT, debug=True)
\end{lstlisting}

\subsection{Initialization of the Database}
\begin{itemize}
  \item \texttt{init\_db()}: Called immediately upon starting the Flask application to ensure that the local SQLite
        database (or whichever database is being used) is properly initialized. This might create tables such as
        \texttt{users} if they do not exist.
\end{itemize}

\subsection{Signup Logic}
\label{subsec:signup}
\begin{itemize}
  \item \textbf{Route:} \texttt{/signup} handles both \texttt{GET} (displaying a signup form) and \texttt{POST} (processing form data).
  \item \textbf{Form Inputs:} Users provide \texttt{email}, \texttt{password}, and \texttt{role} (for example, ``PATIENT'' or ``STUDENT'').
  \item \textbf{add\_user}: A database function that inserts a new record into the \texttt{users} table. If the operation
        is successful, the user is redirected to the login page; otherwise, an error is displayed
        (e.g. ``Email already registered'').
\end{itemize}

\subsection{Login Logic}
\label{subsec:login}
\begin{itemize}
  \item \textbf{Route:} \texttt{/} or \texttt{/login}, handling both \texttt{GET} (showing the login template) and
        \texttt{POST} (verifying credentials).
  \item \textbf{Credential Check:} The user’s provided email is passed to \texttt{get\_user(email)}, which queries the
        database for a matching record (typically returning a tuple \((id, email, password, role)\)).
  \item \textbf{Password Match:} \texttt{user[2] == password} ensures the stored password matches the user’s input.
        If it matches, login is considered successful.
  \item \textbf{Chainlit Process:} After a successful login, the function \texttt{start\_chainlit(email)} is triggered
        in a background thread. This:
        \begin{enumerate}
            \item Checks if Chainlit is already running via \texttt{is\_chainlit\_running()}.
            \item If not running, spawns a subprocess with the user’s email set as an environment variable. This email
                  can then be used by the chatbot to fetch the user’s role from the database without needing URL parameters.
        \end{enumerate}
  \item \textbf{Redirect to Loading Screen:} 
        \begin{itemize}
          \item Renders the template \texttt{loading.html}, which contains a link or auto-redirect to the running Chainlit instance at \texttt{http://localhost:8001}.
        \end{itemize}
  \item \textbf{Error Handling:} If the email/password is incorrect, the user is shown the \texttt{login.html} form
        again, this time with an error message (``Invalid email or password'').
\end{itemize}

\subsection{Role Identification \& Usage}
\begin{itemize}
  \item \textbf{During Signup:} 
  \begin{enumerate}
    \item The \texttt{role} field is selected by the user or assigned automatically (e.g., \texttt{PATIENT} or \texttt{STUDENT}).
    \item \texttt{add\_user} stores the role in the database alongside the \texttt{email} and \texttt{password}.
  \end{enumerate}
  \item \textbf{During Login:}
  \begin{enumerate}
    \item Once the login is validated, the user’s \texttt{role} can be fetched from \texttt{user[3]}. 
    \item In a fully built system, the chatbot interface might reference this role to load different conversation chains
          or specialized functionality. For instance:
          \begin{itemize}
            \item \emph{Medical Students} might get an LLM chain focusing on advanced medical knowledge.
            \item \emph{Patients} might see simplified symptom analysis tools and appointment booking flows.
          \end{itemize}
  \end{enumerate}
\end{itemize}

\subsection{Chainlit Integration}
\label{subsec:chainlit}
\begin{itemize}
  \item \textbf{is\_chainlit\_running()}: Sends an HTTP GET request to \texttt{http://localhost:8001} (the configured Chainlit port).
        If the server responds with a \texttt{200} status code, we assume Chainlit is active.
  \item \textbf{start\_chainlit(email)}: 
        \begin{enumerate}
            \item Uses \texttt{subprocess.Popen} to run \texttt{chainlit} on the given port if not already running.
            \item Exports \texttt{LOGGED\_IN\_EMAIL} as an environment variable. 
                  This allows the Chainlit app to retrieve the user’s role from the database with the same email,
                  applying role-specific logic during chatbot interactions.
        \end{enumerate}
\end{itemize}

\subsection{Overall Flow}
\begin{enumerate}
  \item A new user signs up using \texttt{/signup}, providing email, password, and role.
  \item The system calls \texttt{add\_user} to store these details in the local database.
  \item The user then goes to \texttt{/login}, enters the same email and password, which \texttt{get\_user} verifies.
  \item If successful, \texttt{start\_chainlit} is triggered, launching or reusing a Chainlit session for the user.
  \item The user is redirected to a loading page and then to the Chainlit UI, where their role (retrieved from the database) 
        dictates the AI chatbot’s behavior.
\end{enumerate}

\noindent
In short, this code section ensures:
\begin{itemize}
    \item Secure handling of **login** and **signup** flows.
    \item Automatic **role assignment** and **retrieval** from the database.
    \item Seamless bridging between the Flask web server and the **Chainlit** chatbot interface.
\end{itemize}

\section{PDF Processing (Text Extraction and Chunking)}
\label{sec:pdf-chunking}

In this part of the code, we handle the **user-uploaded PDF files**—from reading the file’s pages to splitting the text into smaller segments (chunks) that are later added to our **vector store** for retrieval. Below is a detailed snippet and explanation:

\begin{lstlisting}[language=Python, caption={PDF Processing and Chunking}, basicstyle=\small\ttfamily]
# If user uploaded a file via paperclip
if message.type == "file":
    logging.info("Processing uploaded file from user via paperclip...")

    # 1 Load the PDF using PyPDFLoader
    loader = PyPDFLoader(message.file_path)
    documents = loader.load()

    # 2 Split text into smaller chunks
    text_splitter = CharacterTextSplitter(chunk_size=4000, chunk_overlap=500)
    texts = text_splitter.split_documents(documents)

    # 3 Add the chunks to the vector store in manageable batches
    batch_size = 10
    for i in range(0, len(texts), batch_size):
        batch = [t.page_content for t in texts[i : i + batch_size]]
        docsearch.add_texts(batch)

    # 4 Notify the user about successful processing
    await cl.Message(
        content=(
            "PDF uploaded and processed successfully! "
            "You can now ask questions about its content."
        )
    ).send()

    return
\end{lstlisting}

\subsubsection*{Detailed Explanation}

\begin{enumerate}
    \item \textbf{Detecting a File Upload}
    \begin{itemize}
        \item The code checks if \texttt{message.type == "file"}. If so, it logs a message indicating that a PDF file has been received from the user through the **paperclip icon** in Chainlit.
        \item Using an \texttt{if} condition around the file-logic ensures text messages and file uploads are handled separately in the same event function.
    \end{itemize}

    \item \textbf{Loading the PDF with \texttt{PyPDFLoader}}
    \begin{itemize}
        \item \texttt{PyPDFLoader} is a document loader that reads PDF files and converts them into a list of \texttt{Document} objects—one for each page. 
        \item Internally, it might use a library like PyPDF2 or pdfplumber to extract text. Here, \texttt{loader.load()} creates an in-memory representation of the PDF content as \texttt{documents}.
    \end{itemize}

    \item \textbf{Splitting Text into Chunks}
    \begin{itemize}
        \item **Why split?** Large text blocks can overwhelm certain embedding or retrieval systems. Breaking down pages into chunks helps keep context manageable and supports more accurate retrieval results (since each chunk remains more focused on a single topic).
        \item \texttt{CharacterTextSplitter}:
        \begin{itemize}
          \item \texttt{chunk\_size=4000} sets the maximum number of characters in a chunk.
          \item \texttt{chunk\_overlap=500} ensures a 500-character overlap between consecutive chunks. This overlap helps preserve continuity of context between two consecutive chunks. When retrieving answers, slightly overlapping text can prevent abrupt context cuts.
        \end{itemize}
        \item \texttt{text\_splitter.split\_documents(documents)}:
        \begin{itemize}
            \item Loops over each \texttt{Document} (e.g., each PDF page) and subdivides it into multiple, partially-overlapping text segments.
            \item Returns a list of \textit{split} text segments, stored in the \texttt{texts} variable. 
        \end{itemize}
    \end{itemize}

    \item \textbf{Batch Insertion into the Vector Store}
    \begin{itemize}
        \item A \textit{vector store} (like Pinecone) is used to store embeddings so that relevant chunks can be quickly retrieved based on semantic similarity.
        \item We define \texttt{batch\_size = 10} to avoid feeding all chunks at once, which could be memory-intensive if the PDF is large.
        \item The loop \texttt{for i in range(0, len(texts), batch\_size)} processes chunks in increments of 10. Each iteration:
        \begin{enumerate}
            \item Slices a \texttt{batch} of 10 from the \texttt{texts} list.
            \item Calls \texttt{docsearch.add\_texts(batch)} to add that subset of chunks to the vector database. 
        \end{enumerate}
        \item Splitting into batches ensures stable performance and reduces the risk of timeouts or memory spikes.
    \end{itemize}

    \item \textbf{User Notification}
    \begin{itemize}
        \item After successfully uploading and splitting the PDF content, the user is informed with a message (``PDF uploaded and processed successfully!''). 
        \item This immediate feedback ensures a good user experience by confirming that the system is ready to handle queries about the newly indexed content.
    \end{itemize}

    \item \textbf{Return Statement}
    \begin{itemize}
        \item The function ends (\texttt{return}) to avoid further processing of the \texttt{message} as if it were a text query. 
        \item This prevents any additional or conflicting logic from executing once the file-handling code finishes.
    \end{itemize}
\end{enumerate}

\subsubsection*{Why Chunking is Important}
\begin{itemize}
  \item \textbf{Efficient Retrieval:} Many LLM-based retrieval systems do best when dealing with smaller segments. Searching and re-ranking short chunks is often faster and more precise.
  \item \textbf{Context Preservation:} Overlapping chunks maintain local context across boundary areas, which is crucial for question answering. If text relevant to a user’s question is split between the end of one chunk and the start of the next, the chunk overlap ensures minimal information loss.
  \item \textbf{Scalability:} Large PDF files with dozens or even hundreds of pages become more manageable. The user can ask questions about any part of the document without manually referencing page numbers or paragraphs.
\end{itemize}

This upload, split, and index workflow is central to enabling a dynamic question-answer system, where each user can bring in new documents and instantly query them. Once the text is converted into vector form, the system can retrieve it for any subsequent queries, seamlessly integrating new medical or academic documents into the user’s knowledge base.

\section{Embedding Generation}
\label{sec:embedding-generation}

This portion of the code lays out how we prepare a sentence-level embedding model and connect it to the vector database—ensuring that new text segments are converted into numerical vectors (embeddings). Below is the relevant snippet, with an in-depth discussion:

\begin{lstlisting}[language=Python, caption={Embedding Model Setup and Pinecone Integration}, basicstyle=\small\ttfamily]
# 1 Load embeddings for all-mpnet-base-v2 (768-dimensional embeddings)
embedding_model = SentenceTransformer('all-mpnet-base-v2')
embeddings = download_hugging_face_embeddings()

# 2 Initialize Pinecone
pinecone_instance = PineconeClient(api_key=PINECONE_API_KEY)
if index_name not in [index.name for index in pinecone_instance.list_indexes()]:
    pinecone_instance.create_index(
        name=index_name,
        dimension=768,
        metric='cosine',
        spec=ServerlessSpec(cloud='aws', region=PINECONE_API_ENV)
    )

# 3 Access the index
docsearch = Pinecone.from_existing_index(index_name, embeddings)
\end{lstlisting}

\subsubsection*{Step-by-Step Explanation}
\begin{enumerate}
    \item \textbf{Loading the Embedding Model}
    \begin{itemize}
        \item \texttt{SentenceTransformer(\textquotesingle all-mpnet-base-v2\textquotesingle)}: 
        We instantiate a pre-trained embedding model from the \emph{Sentence Transformers} family.  
        \begin{itemize}
            \item \emph{all-mpnet-base-v2} outputs vectors of size 768 for each sentence or text chunk.
            \item This particular model is well-regarded for capturing semantic similarities, making it suitable for question answering or retrieval tasks.
        \end{itemize}

        \item \texttt{embeddings = download\_hugging\_face\_embeddings()}: 
        A custom utility, presumably in \texttt{src.helper}, that either fetches or caches embedding-related data.  
        \begin{itemize}
            \item This may ensure version consistency, or optimize performance by not repeatedly downloading large model files.
            \item The result is likely a reference to the same dimension/embedding pipeline used by \texttt{SentenceTransformer}, meaning the code can embed both documents and user queries in a matching format.
        \end{itemize}
    \end{itemize}

    \item \textbf{Initializing Pinecone (Vector Database)}
    \begin{itemize}
        \item \texttt{pinecone\_instance = PineconeClient(api\_key=PINECONE\_API\_KEY)}: 
        This sets up communication with Pinecone’s vector database, using the API key provided via environment variables.
        \item We then check whether our target \texttt{index\_name} (here, \texttt{zf}) exists:
        \begin{itemize}
            \item \texttt{pinecone\_instance.list\_indexes()} returns a list of existing indexes.
            \item If \texttt{zf} is missing, we call \texttt{pinecone\_instance.create\_index(...)}:
            \begin{itemize}
                \item \texttt{dimension=768} aligns with the output dimension of \texttt{all-mpnet-base-v2}.
                \item \texttt{metric='cosine'} ensures we measure distance (or similarity) in terms of cosine similarity, which is a common choice for text embeddings.
                \item \texttt{ServerlessSpec} configures the environment (e.g., \texttt{cloud='aws'}, \texttt{region=PINECONE\_API\_ENV}), controlling where the index is hosted.
            \end{itemize}
        \end{itemize}
        \item If an index called \texttt{zf} already exists, creation is skipped—this prevents overwriting data.
    \end{itemize}

    \item \textbf{Linking the Embedding Model to the Index}
    \begin{itemize}
        \item \texttt{docsearch = Pinecone.from\_existing\_index(index\_name, embeddings)}: 
        We create a \emph{docsearch} object that uses the \texttt{embeddings} logic in tandem with our newly confirmed Pinecone index.
        \item \emph{Why do we pass \texttt{embeddings} here?}  
        This ensures any text we embed for Pinecone is done with the \texttt{all-mpnet-base-v2} model, so that stored vectors and newly embedded queries are directly comparable.
        \item \textbf{Result:} A retrieval interface that can accept text (chunks, user queries, etc.), embed it, and perform similarity lookups within Pinecone.
    \end{itemize}
\end{enumerate}

\subsubsection*{How the Embedding Pipeline Works in Practice}
\begin{enumerate}
    \item \textbf{Ingest Documents:} 
    When the user uploads a PDF (as shown in Section~\ref{sec:pdf-chunking}), each chunk is sent to \texttt{docsearch.add\_texts(...)}. Under the hood, it uses \texttt{embeddings} to transform that chunk into a 768-dimensional vector, then stores it in Pinecone with a unique ID.
    \item \textbf{User Queries:} 
    At query time, the code uses the same embedding model to turn a question into a vector. Pinecone returns the most similar stored chunks (based on cosine similarity).
    \item \textbf{Contextual Answering:} 
    These top chunks (i.e., relevant lines/paragraphs) are then fed into a Large Language Model, which forms a final answer. 
\end{enumerate}

\subsubsection*{Advantages of a Shared Embedding Model}
\begin{itemize}
    \item \textbf{Consistency:} Both documents and queries are embedded using the same approach, guaranteeing semantic alignment.
    \item \textbf{Scalability:} Additional PDFs or text blocks can be seamlessly integrated (embeddings stored) without reprocessing older data.
    \item \textbf{Performance:} Cosine similarity in a vector database like Pinecone is highly optimized, allowing low-latency lookups—even at large scales.
\end{itemize}

\noindent Overall, this embedding generation mechanism is critical for bridging raw textual data with \emph{retrieval-augmented} chat capabilities. Once the embeddings are set up and the index is ready, the system can handle user queries and fetch relevant information in real time.

\section{Retrieval-Augmented Generation Pipeline}
\label{sec:rag-pipeline}

A core feature of this system is its **retrieval-augmented generation (RAG)** approach, where a Large Language Model (LLM) incorporates relevant text chunks from a vector store (Pinecone) to ground its responses in factual context. Below is the relevant code snippet, followed by a detailed walkthrough:

\begin{lstlisting}[language=Python, caption={Retrieval-Augmented Generation (RAG) Pipeline}, basicstyle=\small\ttfamily]
# 9) Long-running QA task
def long_running_task(input_data, llm):
    logging.info("Generating response...")
    qa = RetrievalQA.from_chain_type(
        llm=llm,
        chain_type="stuff",
        retriever=docsearch.as_retriever(search_kwargs={'k': 5}),
        return_source_documents=True,
        chain_type_kwargs={"prompt": PROMPT}
    )
    result = qa.invoke(input_data)
    logging.info("Response generated: %s", result["result"])
    return result["result"]

async_long_running_task = cl.make_async(long_running_task)

# ...

@cl.on_message
async def handle_message(message):
    """
    Main message handler that processes both text messages and file uploads.
    When dealing with user queries, it triggers the RAG pipeline.
    """
    try:
        # ...
        # 1 Distinguish between PDF upload and text queries
        if message.type == "file":
            # [PDF upload handling code removed for brevity]
            return

        # 2 For text messages (i.e., user queries):
        query = message.content.lower()

        # [Optional: Role-based branching omitted here for brevity]
        
        # 3 Re-initialize the same LLM
        llm = initialize_llm()

        # 4 Notify the user that the system is working on the response
        await cl.Message(
            content="Processing your request, please wait..."
        ).send()

        # 5 Perform retrieval
        retriever = docsearch.as_retriever(search_kwargs={'k': 5})
        docs = retriever.invoke(query)
        context = " ".join([doc.page_content for doc in docs])

        # 6 Prepare the combined input data for the chain
        input_data = {"query": query, "context": context}

        # 7 Execute the retrieval-augmented QA pipeline asynchronously
        result = await async_long_running_task(input_data, llm)

        # 8 Send the final answer back to the user
        await cl.Message(content=result).send()

    except Exception as e:
        logging.error(f"Error occurred: {e}")
        await cl.Message(
            content="An error occurred while processing your request. Please try again."
        ).send()
\end{lstlisting}

\subsubsection*{Detailed Explanation}
\begin{enumerate}
    \item \textbf{Long-running QA Task (\texttt{long\_running\_task})}
    \begin{itemize}
        \item We define a function that bundles up \emph{Retrieval + Generation} into a single workflow:
        \begin{itemize}
            \item \texttt{RetrievalQA.from\_chain\_type(...) }: A LangChain method that composes two main elements:
                \begin{enumerate}
                    \item A \emph{retriever}, which fetches the most relevant chunks (here, using \texttt{docsearch.as\_retriever()})  
                    \item A language model (\texttt{llm}) that we pass in as a parameter.  
                \end{enumerate}
            \item \texttt{chain\_type="stuff"}: Tells LangChain to “stuff” the retrieved chunks into a single prompt, which the LLM then processes.
            \item \texttt{return\_source\_documents=True}: Ensures the pipeline can return which text chunks were used, aiding traceability and debug.
            \item \texttt{chain\_type\_kwargs=\{"prompt": PROMPT\}}: Injects our custom prompt template, controlling how context and user query are combined in the final prompt.
        \end{itemize}
        \item \texttt{result = qa.invoke(input\_data)}: Takes the dict \{\texttt{"query"}, \texttt{"context"}\} and feeds it into the chain. The chain formats a prompt (via \texttt{PROMPT}), appends relevant text, and runs the final LLM call.
        \item The function logs the output and returns \texttt{result["result"]}, the LLM’s textual answer.
    \end{itemize}

    \item \textbf{Asynchronous Execution (\texttt{async\_long\_running\_task})}
    \begin{itemize}
        \item \texttt{cl.make\_async} wraps the synchronous \texttt{long\_running\_task} so it can be awaited within an \texttt{async} function (\texttt{handle\_message}). 
        \item This prevents the entire chatbot loop from blocking while the LLM generates a response—especially important if inference is slow or if many users are connected.
    \end{itemize}

    \item \textbf{Main Message Handler (\texttt{handle\_message})}
    \begin{itemize}
        \item This function processes every user message. It distinguishes between file uploads (\texttt{if message.type == "file"}) and text queries.
        \item If it’s a \emph{text query}, the following steps occur:
        \begin{enumerate}
            \item \texttt{query = message.content.lower()}: The user’s text is captured.
            \item \texttt{llm = initialize\_llm()}: We load or re-initialize the LLM (e.g., a Mistral model). 
            \item \texttt{await cl.Message(content="...").send()}: A quick message is sent to let the user know the system is working.
            \item \textbf{Retrieval}: 
            \begin{itemize}
                \item \texttt{retriever = docsearch.as\_retriever(search\_kwargs=\{'k': 5\})} uses the Pinecone-based docsearch to find up to five relevant chunks for the user’s query.
                \item \texttt{docs = retriever.invoke(query)} runs the similarity lookup. 
                \item Then we build a \texttt{context} string out of these chunks, typically by concatenating them with a space.
            \end{itemize}
            \item \textbf{Chain Input}: 
            \begin{itemize}
                \item \texttt{input\_data = \{"query": query, "context": context\}} is the dictionary that gets passed to our QA chain (via \texttt{long\_running\_task}).
            \end{itemize}
            \item \texttt{result = await async\_long\_running\_task(input\_data, llm)}: We run the RAG pipeline asynchronously.
            \item Finally, we send the LLM’s answer back to the user with \texttt{await cl.Message(content=result).send()}.
        \end{enumerate}
    \end{itemize}

    \item \textbf{Prompt Template (\texttt{PROMPT})}
    \begin{itemize}
        \item Defined elsewhere in the code, it might look like:
\begin{lstlisting}[language=Python]
PROMPT = PromptTemplate(
    template=(
        "You are a medical assistant. "
        "Based on the following context, answer the question concisely:\n\n"
        "Context: {context}\n\n"
        "Question: {question}\n\n"
        "Answer:"
    ),
    input_variables=["context", "question"]
)
\end{lstlisting}
        \item The placeholders \texttt{{context}} and \texttt{{question}} are substituted with the text from the retriever. 
        \item This structure forces the LLM to ground its answer in the retrieved snippets, mitigating hallucinations.
    \end{itemize}
\end{enumerate}

\subsubsection*{Why RAG Matters}
\begin{itemize}
    \item \textbf{Accuracy and Context}: The LLM is less likely to invent facts, since it sees the actual text relevant to the user’s question.
    \item \textbf{Extensibility}: Users can add more documents over time. The system can then answer questions about newly introduced material without retraining the LLM.
    \item \textbf{Traceability}: Because \texttt{return\_source\_documents=True} is set, the pipeline can record which chunks were used, enabling better debugging or even letting the user see those sources for reference.
\end{itemize}

\noindent
Overall, **retrieval-augmented generation** fuses an \textit{external knowledge store} (Pinecone) with a powerful \textit{language model} (like Mistral). The final outcome is a **contextual, verifiable answer** that draws from user-uploaded PDFs or other indexed material, rather than relying solely on the LLM’s internal training data.

\section{Role-Based Logic (Learning Assistant vs. Diagnostic Agent)}
\label{sec:role-logic}

In this system, the user’s \emph{role}—either a \textbf{Student} in need of detailed, academic explanations or a \textbf{Patient} requiring concise diagnostic guidance—shapes the chatbot’s behavior. Below is a relevant code snippet from the \texttt{start\_chat} and \texttt{handle\_message} functions, illustrating how we differentiate between these modes:

\begin{lstlisting}[language=Python, caption={Role-Based Logic Snippet}, basicstyle=\small\ttfamily]
@cl.on_chat_start
async def start_chat():
    """
    On chat start, we read LOGGED_IN_EMAIL from environment,
    fetch the user's role from DB, store it in session, and greet them.
    """
    email = os.environ.get("LOGGED_IN_EMAIL", "")
    role = None

    if email:
        user = get_user(email)  # e.g. (id, email, password, role)
        if user:
            role = user[3].strip().upper()  # e.g. 'PATIENT' or 'STUDENT'

    cl.user_session.set("role", role)

    if role == "PATIENT":
        await cl.Message(
            content=(
                "Patient Chatbot\n\n"
                "Welcome! You can upload a PDF (via the paperclip icon) or ask a biomedical question. "
                "To book an appointment, type 'book an appointment'."
            )
        ).send()
    elif role == "STUDENT":
        await cl.Message(
            content=(
                "Student Chatbot\n\n"
                "Welcome! You can upload a PDF (paperclip icon) or ask a biomedical question to get started."
            )
        ).send()
    else:
        await cl.Message(content="ERROR: Role not recognized. Contact Support.").send()

@cl.on_message
async def handle_message(message):
    try:
        role = cl.user_session.get("role")  # 'PATIENT' or 'STUDENT' or None

        # If user is a PATIENT, handle special logic (e.g., appointment booking)
        if role == "PATIENT":
            # [omitted: user_appointment_info logic, location/specialty checks, etc.]

            # If no special triggers, continue to normal query handling
            # ...

        # If user is a STUDENT, or we are done with appointment logic:
        # ... (common retrieval + LLM steps go here)

    except Exception as e:
        logging.error(f"Error occurred: {e}")
        await cl.Message(
            content="An error occurred while processing your request. Please try again."
        ).send()
\end{lstlisting}

\subsubsection*{Detailed Explanation}

\begin{enumerate}
    \item \textbf{Role Extraction \& Storage}
    \begin{itemize}
        \item During \texttt{start\_chat}, the code retrieves the user’s email from environment variables (in this example, \texttt{LOGGED\_IN\_EMAIL}) and queries the database via \texttt{get\_user} to find the stored role (e.g., \texttt{'PATIENT'} or \texttt{'STUDENT'}).
        \item \texttt{cl.user\_session.set("role", role)} saves this role in the session context, ensuring that each subsequent message from the user automatically references the same role without needing repeated lookups.
    \end{itemize}

    \item \textbf{Greeting the User}
    \begin{itemize}
        \item If the role is \texttt{PATIENT}, the chatbot introduces itself as a “Patient Chatbot,” highlighting PDF upload and biomedical question features. It also mentions the logic for appointment booking (i.e., “book an appointment”).
        \item If the role is \texttt{STUDENT}, the chatbot welcomes them with a focus on deeper academic or learning-related inquiries, referencing file uploads for coursework or research documents.
        \item If the system cannot identify a valid role, it displays an error message advising the user to contact support.
    \end{itemize}

    \item \textbf{Message Handling \& Conditional Flows}
    \begin{itemize}
        \item In \texttt{handle\_message}, we re-check the user’s role from the \texttt{cl.user\_session}.
        \item \textbf{If \texttt{role == "PATIENT"}}, specialized appointment logic is available:
            \begin{itemize}
                \item For instance, searching for “book an appointment” triggers additional steps, such as collecting the user’s city and specialty, or sending a Twilio SMS with a booking link.
                \item If no special triggers are detected, the code falls back to normal retrieval+LLM query handling.
            \end{itemize}
        \item \textbf{If \texttt{role == "STUDENT"}}, the user receives more in-depth, academically oriented assistance (e.g., expanded explanations, references).
        \item In both cases, after role-specific checks, we eventually proceed to the standard retrieval pipeline for general Q\&A or synergy with the LLM.
    \end{itemize}
\end{enumerate}

\subsubsection*{Why Role-Based Logic?}

\begin{itemize}
    \item \textbf{Differentiated Interaction}: Patients have simple, symptom-focused questions and possibly need appointment scheduling. Meanwhile, students want detailed, academically rigorous explanations.
    \item \textbf{Security \& Personalization}: Certain features, such as patient data or contact info, should not be accessible to a “student” user. Role-based checks help maintain the appropriate scope of functionalities.
    \item \textbf{Modular Scalability}: Additional roles (e.g., “Doctor,” “Researcher,” or “Admin”) could be integrated in the future, each with specialized logic or expanded capabilities.
\end{itemize}

\noindent
Overall, role-based logic ensures the chatbot adapts to the user’s specific needs. Patients can focus on receiving quick, practical medical guidance, while students can explore in-depth academic or biomedical topics—all handled smoothly in a single application.

\section{Symptom Analysis, Provider Recommendation, and Appointment Booking}
\label{sec:provider-appointment}

Beyond general Q\&A functionality, the system also handles patient-centric tasks: analyzing user symptoms, 
recommending healthcare providers, and even automating appointment bookings. Below is an excerpt illustrating 
how these features appear in the \texttt{handle\_message} function for \texttt{role == "PATIENT"}:

\begin{lstlisting}[language=Python, caption={Symptom Analysis \& Appointment Booking Snippet}, basicstyle=\small\ttfamily]
# ...
if role == "PATIENT":
    query = message.content.lower()

    # 1 Trigger for booking an appointment
    if "book an appointment" in query:
        user_appointment_info[message.author] = {}
        await cl.Message(
            content="Sure! What is your location (e.g., Berlin, Munich)?"
        ).send()
        return

    # 2 If we already started booking logic
    if message.author in user_appointment_info:
        if "location" not in user_appointment_info[message.author]:
            # Capture patients location
            user_appointment_info[message.author]["location"] = message.content
            await cl.Message(
                content="Got it! What specialty do you need (e.g., cardiology, dermatology)?"
            ).send()
            return

        elif "specialty" not in user_appointment_info[message.author]:
            # Capture specialty
            user_appointment_info[message.author]["specialty"] = message.content
            await cl.Message(
                content="Finally, please provide your phone number so we can send the booking link."
            ).send()
            return

        elif "phone_number" not in user_appointment_info[message.author]:
            # 3 Final step: generate & SMS booking link
            user_appointment_info[message.author]["phone_number"] = message.content
            
            # Convert city & specialty to German equivalents
            city_raw = user_appointment_info[message.author]["location"].strip()
            specialty_raw = user_appointment_info[message.author]["specialty"].strip()
            # ... (use translation maps) ...
            
            booking_link = f"https://www.doctolib.de/{specialty_converted}/{city_converted}"
            twilio_client.messages.create(
                body=f"Your appointment booking link: {booking_link}",
                from_=TWILIO_PHONE_NUMBER,
                to=user_appointment_info[message.author]["phone_number"]
            )
            await cl.Message(
                content=f"Your booking link has been sent to {message.content}."
            ).send()
            # Clear user_appointment_info
            del user_appointment_info[message.author]
            return

    # 4 Symptom or general patient queries fall back to normal retrieval + LLM
    # ...
\end{lstlisting}

\subsubsection*{Detailed Explanation}

\begin{enumerate}
    \item \textbf{Triggering Appointment Booking}
    \begin{itemize}
        \item If a patient’s query includes the phrase \texttt{"book an appointment"}, the system captures their user ID (i.e., \texttt{message.author}) in a dictionary called \texttt{user\_appointment\_info}.
        \item This dictionary then tracks key steps (location, specialty, phone number) needed to finalize an appointment.
        \item A message is immediately returned asking for the user’s location, preventing further text processing for that round.
    \end{itemize}

    \item \textbf{Location \& Specialty Collection}
    \begin{itemize}
        \item Once the user flows into the “appointment” path, the logic checks which pieces of information are missing:
        \begin{enumerate}
            \item \texttt{location}: e.g., Berlin or Munich
            \item \texttt{specialty}: e.g., Cardiology, Dermatology
        \end{enumerate}
        \item The code displays short messages at each step. This ensures the user is guided through a simplified multi-turn conversation.
    \end{itemize}

    \item \textbf{Phone Number \& Booking Link Generation}
    \begin{itemize}
        \item When the phone number is received, the code triggers a final step that:
        \begin{enumerate}
            \item Converts the city name and specialty into a German-compatible format via a dictionary lookup (\texttt{city\_translation\_map}, \texttt{translation\_map}).
            \item Constructs a \texttt{booking\_link} to a medical appointment platform (e.g., Doctolib).
            \item Sends the link via \texttt{twilio\_client.messages.create(...)}, delivering an SMS to the user.
            \item Responds in chat with a confirmation message (``Your booking link has been sent...''), then removes the user from \texttt{user\_appointment\_info} to avoid confusion on future messages.
        \end{enumerate}
    \end{itemize}

    \item \textbf{Symptom or General Queries}
    \begin{itemize}
        \item For patient queries that aren’t directly about booking (e.g., “I have a sore throat...”), the code can fall back to normal retrieval + LLM logic.
        \item This might involve symptom analysis or retrieving from a relevant knowledge base. The snippet above omits that for brevity.
    \end{itemize}
\end{enumerate}

\subsubsection*{Provider Recommendation and Symptom Analysis}

\begin{itemize}
    \item \textbf{Provider Recommendation:}
    \begin{itemize}
        \item The specialized dictionaries (\texttt{city\_translation\_map}, \texttt{translation\_map}) map English city or specialty names to their German equivalents. This ensures the final booking URL matches the local healthcare provider listings.
        \item Further expansions could include an internal or external “doctor directory,” letting the chatbot automatically suggest providers based on location and specialty, even before generating a link.
    \end{itemize}
    \item \textbf{Symptom Analysis:}
    \begin{itemize}
        \item If the user states symptoms (e.g., “I have frequent headaches”), the code can direct the user to appropriate follow-up logic or mention relevant specialties.  
        \item In some advanced setups, an LLM prompt specifically includes “symptom triage” or references medical guidelines, bridging from text-based retrieval to real-time suggestions (the code snippet is mostly an outline for appointment logic).
    \end{itemize}
\end{itemize}

\noindent
Overall, this patient-focused portion of the system ensures that common tasks—like clarifying symptoms, picking the right specialist, and securing an appointment—are streamlined. By combining manual user input flow with automated translation logic and an SMS-based booking confirmation, the bot delivers a practical end-to-end healthcare assistance workflow.

\section{Front-End Integration}
\label{sec:frontend}

While much of the system’s logic operates on the back end (via Flask and LangChain/Chainlit), the **front-end layer** is critical for user interaction. This section describes how the UI for login, signup, and Chainlit chat seamlessly tie into the back-end logic.

\subsection{Login and Signup Pages (Flask Templates)}

Below is an example of how we might structure the \texttt{login.html} and \texttt{signup.html} pages. They rely on **Flask**’s \texttt{render\_template} mechanism:

\begin{lstlisting}[language=HTML, caption={login.html (Flask Template)}, basicstyle=\small\ttfamily]
<!DOCTYPE html>
<html>
<head>
    <title>Login</title>
</head>
<body>
    <h2>Login</h2>
    
        <p style="color:red;">{{ error }}</p>
    
    <form method="POST" action="/">
        <label for="email">Email:</label>
        <input type="text" name="email" required><br><br>
        
        <label for="password">Password:</label>
        <input type="password" name="password" required><br><br>

        <button type="submit">Login</button>
    </form>
    <p>
        Don't have an account? 
        <a href="{{ url_for('signup') }}">Sign up here</a>
    </p>
</body>
</html>
\end{lstlisting}

\noindent\textbf{Explanation:}
\begin{itemize}
    \item The page includes a simple form posting to \texttt{/} (the login route).
    \item If there’s an \texttt{error} (set by Flask upon invalid credentials), it displays in red text.
    \item A link leads users to the \texttt{signup} route if they have no account.
\end{itemize}

\begin{lstlisting}[language=HTML, caption={signup.html (Flask Template)}, basicstyle=\small\ttfamily]
<!DOCTYPE html>
<html>
<head>
    <title>Signup</title>
</head>
<body>
    <h2>Create an Account</h2>
    
        <p style="color:red;">{{ error }}</p>
    
    <form method="POST" action="/signup">
        <label for="email">Email:</label>
        <input type="text" name="email" required><br><br>
        
        <label for="password">Password:</label>
        <input type="password" name="password" required><br><br>

        <label for="role">Role (PATIENT / STUDENT):</label>
        <input type="text" name="role" required><br><br>

        <button type="submit">Sign Up</button>
    </form>
    <p>
        Already have an account? 
        <a href="{{ url_for('login') }}">Login here</a>
    </p>
</body>
</html>
\end{lstlisting}

\noindent\textbf{Explanation:}
\begin{itemize}
    \item This form posts to \texttt{/signup}.
    \item The user specifies an \texttt{email}, \texttt{password}, and \texttt{role} (e.g., “PATIENT” or “STUDENT”).
    \item If the email is already taken, an \texttt{error} message (like ``Email already registered.'') is shown.
\end{itemize}

\subsection{Chainlit Chat UI}

\textbf{Chainlit} provides a sleek front-end interface out of the box. After the user logs in (and the Flask code launches or reuses a Chainlit instance), the user is redirected to the Chainlit UI, typically at \texttt{http://localhost:8001}. Some key features of the Chainlit interface:

\begin{itemize}
    \item **Interactive Chat Windows**: Users can type or upload PDFs (via the \emph{paperclip} icon) directly. 
    \item **Session Awareness**: The environment variable \texttt{LOGGED\_IN\_EMAIL} and the role from the database guide how the chatbot greets or handles user messages.
    \item **Chat Profiles**: As defined in \texttt{@cl.set\_chat\_profiles}, we can brand the chat or provide special instructions (e.g., “Mistral Biomedical” for medical queries).
    \item **Real-time Updates**: The user sees status messages (like “Processing your request...”) and final LLM replies in real time.
\end{itemize}

\subsubsection{Loading Page (Optional)}

A simple \texttt{loading.html} can serve as a bridge page after successful login, informing the user that **Chainlit** is starting up. For instance:

\begin{lstlisting}[language=HTML, caption={loading.html (Flask Template)}, basicstyle=\small\ttfamily]
<!DOCTYPE html>
<html>
<head>
    <title>Loading Chatbot...</title>
</head>
<body>
    <h2>Please wait, we are launching your chatbot session...</h2>
    <p>If not redirected automatically, <a href="{{ chatbot_url }}">click here</a>.</p>
</body>
</html>
\end{lstlisting}

\noindent
Flask’s login route references this by calling \texttt{render\_template("loading.html", chatbot\_url=chatbot\_url)}.

\subsection{Overall Flow}

\begin{enumerate}
    \item \textbf{User Visits Login Page (\texttt{/})}:
    \begin{itemize}
        \item If they **POST** valid credentials, the server spawns or reuses a Chainlit process and redirects them to \texttt{loading.html}.
    \end{itemize}
    \item \textbf{Loading Screen}:
    \begin{itemize}
        \item The user is shown a simple “please wait” message. After a short wait, they can click (or get auto-redirected) to \texttt{http://localhost:8001}.
    \end{itemize}
    \item \textbf{Chainlit UI}:
    \begin{itemize}
        \item The front-end chat interface greets the user based on their role (Patient / Student).
        \item Users can type questions or click the paperclip to upload a PDF. 
        \item Each message is sent to the back-end event handlers, which respond with LLM-driven answers or further instructions (e.g., location prompts for appointment booking).
    \end{itemize}
\end{enumerate}

\noindent
By combining **Flask** for authentication and session management with **Chainlit** for conversational interaction, we achieve a user-friendly, cohesive experience:
\begin{itemize}
    \item **Flask**: Manages the database (login, signup, roles) and starts the Chainlit server if needed.
    \item **Chainlit**: Provides the interactive “chat window,” letting users upload documents, ask questions, and receive LLM-powered responses in real time.
\end{itemize}


\end{document}


\end{document}
